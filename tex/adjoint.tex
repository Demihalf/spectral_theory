\section{Сопряжённые операторы}

Далее будет рассматриваться конечномерное пространство со скалярным
произведением (евклидово пространство) $H$ над полем
действительных либо комплексных чисел.

\begin{lemma} \label{le:functionalform}
    Любой линейный функционал $\xi : H \to \fieldk$ допускает единственное
    представление вида
    \[ \xi(x) = \langle x, a\rangle, \quad a \in H \]
\end{lemma}

\begin{proof}
    Выберем \hspace{0.1em} некоторый \hspace{0.05em} ортонормированный базис $e_1, \dotsc, e_n$ в $H$. Пусть $\xi : H \to
    \fieldk$ — линейный функционал. Определим $a$ следующим образом:
    \[ a = \sum_{j=1}^n \overline{\xi(e_j)} e_j \]

    Рассмотрим скалярное произведение $\langle e_k, a \rangle, \; k =
    \overline{1,n}$:
    \[ \langle e_k, a \rangle = \langle e_k, \sum_{j=1}^n \overline{\xi(e_j)}
    e_j \rangle = \langle e_k, \overline{\xi(e_k)} e_k \rangle = \xi(e_k) \]

    Тогда
    \[ \xi(x) = \sum_{j=1}^n x_j \xi(e_j) = \sum_{j=1}^n x_j \langle
        e_j, a\rangle = \langle \sum_{j=1}^m x_j e_j, a\rangle = \langle
    x, a\rangle \] 
\end{proof}

\begin{definition}
    Оператор $B\in L(H)$ называется \emph{сопряжённым к} $A\in L(H)$, если
    \[ \langle Ax, y \rangle = \langle x, By\rangle \quad \forall x, y \in H\]
\end{definition}

\begin{lemma}
    Для любого \hspace{1mm} линейного оператора \hspace{1mm} в евклидовом пространстве существует
    единственный сопряженный оператор.
\end{lemma}

\begin{proof}
    Пусть $y \in H$. 
    
    Рассмотрим функционал $\xi_y : H \to \fieldk$ вида
    \[ \xi_y(x) = \langle Ax, y\rangle \]

    По лемме 1 существует $a \in H$, что 
    \[ \xi_y(x) = \langle x, a\rangle \]

    Положим $By = a$. Докажем, что $B$ — линейный оператор из $L(H)$.

    Пусть $y = \alpha y_1 + \beta y_2$.
    \begin{gather*}
        \begin{alignedat}{2}
            \xi_{y_1}(x) &= \langle Ax, y_1\rangle &&= \langle x, By_1\rangle \\
            \xi_{y_2}(x) &= \langle Ax, y_2\rangle &&= \langle x, By_2\rangle \\
            \xi_{y}(x) &= \langle Ax, \alpha y_1 + \beta y_2 \rangle &&= \langle x,
            B(\alpha y_1 + \beta y_2) \rangle
        \end{alignedat} \\
        \begin{multlined}
            \xi_{y}(x) = \langle Ax, \alpha y_1 \rangle + \langle Ax, \beta y_2
            \rangle = \overline{\alpha}\xi_{y_1}(x) + \overline{\beta}\xi_{y_2}(x) =\\= \langle x, \alpha
            By_1 + \beta By_2\rangle
        \end{multlined} \\
        \langle x, B(\alpha y_1 + \beta y_2) \rangle = \langle x, \alpha By_1 +
        \beta By_2\rangle \\
        B(\alpha y_1 + \beta y_2) = \alpha By_1 + \beta By_2 \\
        \langle Ax, y\rangle = \langle x, By\rangle
    \end{gather*}
\end{proof}

Сопряженный оператор обозначается звёздочкой:
\[ A^* = B \]

\begin{lemma}[свойства сопряженных операторов]\label{lemma:adjointproperties}
    Для \\ любых операторов $A, B \in L(H)$ и любых чисел $\alpha, \beta \in
    \fieldk$ справедливы следующие утверждения:
    \begin{enumerate}
        \item $ (\alpha I)^* = \overline{\alpha}I $
        \item $ (AB)^* = B^* A^* $
        \item $ (\alpha A + \beta B)^* = \overline{\alpha} A^* +
            \overline{\beta} B^* $
        \item Если $A$ обратим, то и $A^*$ обратим, причём
            \[ (A^{-1})^* = (A^*)^{-1} \]
        \item $ (A^*)^* = A $
        \item Если $M$ — инвариантное подпространство $A$ и $A^*$, то 
            \[ (A|M)^* = A^* | M \]
    \end{enumerate}
\end{lemma}

\begin{proofbreak}
    \begin{enumerate}
        \item $
                \langle \alpha I x, y \rangle = \alpha \langle x, y\rangle =
                \langle x, \overline{\alpha} y \rangle = \langle x,
                \underbracket[0.5pt]{\overline{\alpha} I}_{\mathclap{=(\alpha I)^*}} y\rangle    
                $
        \item $ \langle ABx, y \rangle = \langle A(Bx), y\rangle = \langle
            Bx, A^*y\rangle = \langle x, B^*A^*y\rangle $
        \item Доказывается аналогично (используются свойства скалярного
            произведения)
        \item $
            \begin{alignedat}[t]{2}
                AA^{-1} &= A^{-1}A &&= I \\
                (AA^{-1})^* &= (A^{-1}A)^* &&= I^* \\
                (A^{-1})^* A^* &= A^* (A^{-1})^* &&= I \\
                (A^{-1})^* &= (A^*)^{-1}
            \end{alignedat}
            $
        \item Доказывается при помощи свойства эрмитовой симметричности скалярного
            произведения.
        \item Очевидно.
    \end{enumerate}
\end{proofbreak}

\begin{lemma}\label{lemma:orthogonaladjoint}
    Пусть $A \in L(H)$. Тогда $H$ раскладывается в ортогональную прямую сумму
    подпространств $H = \im A \oplus \ker A^*$ и $H = \ker A \oplus \im A^*$.
\end{lemma}

\begin{proof}
    Для доказательства леммы достаточно показать, что $(\im A)^\perp = \ker
    A^*$. Пусть $y \in (\im A)^\perp$. Тогда $\langle Ax, y\rangle = 0$, т. к.
    $\forall x \in H \; y \perp Ax \in \im A$. Значит $\langle x, A^*y\rangle =
    0$. Тогда, в силу произвольности $x$, подставляя вместо него $A^* y$
    получаем
    \[ \langle A^* y, A^* y\rangle = 0 \Rightarrow A^* y = 0 \Rightarrow y \in
    \ker A^* \]

    Теперь обратно: пусть $y \in \ker A^*$. Тогда $\langle x, A^* y\rangle = 0$,
    и $\langle Ax, y\rangle = 0$, значит $y \perp \im A$ и тогда $y \in (\im
    A)^\perp$.
\end{proof}

Непосредственно из леммы \ref{lemma:orthogonaladjoint} следует
\begin{theorem}[Фредгольма]
    Уравнение вида
    \[ Ax = b, \quad b \in H \]
    разрешимо тогда и только тогда, когда вектор $b$ перпендикулярен всем
    решениям однородного уравнения
    \[ A^*y = 0 \]
\end{theorem}

\begin{lemma}\label{lemma:adjointmatrix}
    Пусть $e_1, \dotsc, e_n$ \, — \, ортонормированный базис, $A, B \in L(H)$,
    $\mathcal{A} = (a_{ij}) \in \matr{n}$, $\mathcal{B} = (b_{ij}) \in \matr{n}$
    — матрицы этих операторов в том же базисе. Тогда для того чтобы $B = A^*$
    необходимо и достаточно, чтобы $b_{ij} = \overline{a_{ji}}$.
\end{lemma}

\begin{proof}
    \[ Ae_j = x = \sum_{i=1}^n a_{ij}e_i = \sum_{i=1}^n \langle x, e_i\rangle
    e_i = \sum_{i=1}^n \langle Ae_j, e_i \rangle \]

    Таким образом 
    \[ a_{ij} = \langle Ae_j, e_i\rangle \]

    Доказательство в обе стороны проводится аналогично:
    \begin{gather*}
            Ae_j = \sum_{i=1}^n a_{ij} e_i \\
            a_{ij} = \langle Ae_j, e_i \rangle = \langle e_j, A^* e_i\rangle =
            \overline{\langle A^* e_i, e_j\rangle} = \overline{b_{ji}}
    \end{gather*}
\end{proof}

\begin{definition}
    Оператор $A\in L(H)$ называется \emph{самосопряженным}, если $A^* = A$
    \[ \langle Ax, y\rangle = \langle x, Ay\rangle \]
\end{definition}

\begin{example}
    Пусть $A\in L(H)$. Тогда операторы
    \begin{align*}
        \Re A &= \frac{A + A^*}{2} \\
        \Im A &= \frac{A - A^*}{2i}
    \end{align*}
    являются самосопряженными.  При этом $A = \Re A + i \Im A$.
\end{example}

\begin{example}
    Самосопряженным является оператор $AA^*$:
    \[ (AA^*)^* = (A^*)^* A^* = AA^* \]
\end{example}

\begin{definition}
    Матрица $\mathcal{B} = (b_{ij}) \in \matr{n}$ называется сопряженной к
    $\mathcal{A} = (a_{ij}) \in \matr{n}$, если
    \[ b_{ij} = \overline{a_{ji}} \]
\end{definition}

Непосредственно из леммы \ref{lemma:adjointmatrix} следует, что оператор $A\in
L(H)$ является самосопряженным тогда и только тогда, когда его матрица в
некотором ортонормированном базисе является самосопряженной.

\begin{definition}
    Проектор $P\in L(H)$\, называется \emph{ортогональным (ортопроектором)}, если
    он осуществляет разложение $H$ в ортогональную прямую сумму.
    \begin{gather*}
        H = \im P \oplus \im (I - P) \\
        \im P \perp \im (I - P) = \ker P
    \end{gather*}
\end{definition}

\begin{lemma}
    Проектор $P\in L(H)$ является ортопроектором тогда и только тогда, когда
    $P^* = P$.    
\end{lemma}

\begin{proofbreak}
    \dindent \textbf{Необходимость}

    Пусть $P$ — ортопроектор, то есть
    \[ \im P \perp \im (I - P) \]

    Пусть $H \ni x = x_1 + x_2$, $H \ni y = y_1 + y_2$, где $x_1, y_1 \in \im
    P$, $x_2, y2 \in \im (I - P$.
    \begin{multline*}
        \langle Px, y\rangle = \langle x_1, y_1 + y_2 \rangle = 
        \langle x_1, y_1 \rangle + \langle x_1, y_2 \rangle =\\= 
        \langle x_1, Py \rangle = \langle x_1 + x_2, Py
        \rangle = \langle x, Py \rangle
    \end{multline*}

    \textbf{Достаточность}

    Пусть проектор $P$ самосопряжен. Тогда по лемме
    \ref{lemma:orthogonaladjoint} получаем, что 
    \[ H = \im P \oplus \ker P = \im P \oplus \im (I - P) \]
    где предполагается разложение в ортогональную прямую сумму.
\end{proofbreak}

\begin{corollaryle}
    Дополнительный проектор \hspace{0.1em} к ортопроектору есть ортопроектор.
    \[ (I - P)^* = I^* - P^* = I - P \]
\end{corollaryle}

\begin{definition}
    $B \in L(H)$ называется \emph{антисамосопряженным (кососамосопряженным)},
    если $B^* = -B$.
\end{definition}

\begin{lemma}
    Если $B \in L(H)$ — антисамосопряженный оператор, то существует
    самосопряженный оператор $A$ такой, что $B = iA$.    
\end{lemma}

\begin{proof}
    Пусть $B^* = -B$, $A = -i B$.
    \[ A^* = i B^* = -iB = A \]
    то есть $A$ — самосопряженный оператор и $B = iA$.
\end{proof}

\begin{definition}
    Оператор $U \in L(H)$ называется \emph{унитарным}, если имеют место равенства:
    \[ U^* U = UU^* = I, \]
    т. е. $U$ — обратим и $U^* = U^{-1}$.
\end{definition}

Пусть $U$ — унитарный оператор.
\[ \norm{Ux}^2 = \langle Ux, Ux \rangle = \langle x, U^* Ux\rangle = \langle
x, x \rangle = \norm{x}^2 \]

То есть \emph{всякий унитарный оператор сохраняет длины векторов (является
изометрией)} (можно показать, что верно и обратное: всякий изометрический
изоморфизм есть унитарный оператор).

\begin{definition}
    Оператор $A \in L(H)$ называется \emph{нормальным}, если
    \[ AA^* = A^* A \]
\end{definition}

Непосредственно из определений следует, что самосопряженные, кососамосопряженные
и унитарные операторы являются нормальными.

\begin{lemma}\label{lemma:invariantcommute}
    Пусть $A, B \in L(X)$ — операторы в произвольном линейном пространстве — перастановочны:
    \[ AB = BA \]

    Тогда $\ker B$ и $\im B$ являются инвариантными подпространствами для
    $A$
\end{lemma}

\begin{proof}
    Пусть $x \in \ker B$, тогда $Bx = 0$
    \[ B(Ax) = A(Bx) = A0 = 0 \Rightarrow Ax \in \ker B \]

    Аналогично доказывается для образа.
\end{proof}

\begin{lemma}
    \begin{equation*}
        \spectrum{A^*} = \overline{\spectrum{A}}
    \end{equation*}
\end{lemma}

\begin{proof}
    Пусть $\lambda \in \fieldk$.
    \[ (A - \lambda I)^* = A^* - \overline{\lambda} I \]
    
    Оператор $A - \lambda I$ обратим тогда и только тогда, когда обратим $A^* -
    \overline{\lambda}I$, откуда и следует утверждение теоремы.
\end{proof}

\begin{theorem}\label{th:normaldiagonalizable}
    Пусть $A \in L(H)$ — нормальный оператор, его спектр $\spectrum{A} = \menge{\lambda_1,
    \dotsc, \lambda_m}$, $\fieldk = \fieldc$. Тогда
    \begin{enumerate}
        \item $E(\overline{\lambda_i}, A^*) = E(\lambda_i, A), \; i =
            \overline{1,m}$
        \item Пространство $H$ раскладывается в ортогональную прямую сумму
            \[ H = E(\lambda_1, A) \oplus \dotsb \oplus E(\lambda_m, A) \]
        \item $A$ и $A^*$ — операторы простой структуры
    \end{enumerate}
\end{theorem}

\begin{proofbreak}
    \dindent Рассмотрим собственное подпространство $E(\lambda_k, A) = M$. Поскольку $A$ и
    $A^*$ перестановочны, то по лемме \ref{lemma:invariantcommute} для оператора
    $A^*$ подпространство $M$ также будет инвариантным.

    Покажем, что $E(\lambda_k, A) \subset E(\overline{\lambda_k},A^*)$.
    Пусть $x \in E(\lambda_k, A)$.
    Рассмотрим сужение $(A-\lambda_k I)_M = O_M$. $M$ инвариантно для
    $A-\lambda_k I$ и для $A^* - \overline{\lambda_k} I$ и по пункту (6) леммы
    \ref{lemma:adjointproperties} получаем, что
    \[ O_M = (A-\lambda_k I)_M = ((A-\lambda_k)^*)_M = (A^* - \overline{\lambda_k}
    I)_M \]

    Таким образом
    \[ (A^* - \overline{\lambda_k} I)_M = O_M \]

    Значит $A^*x = \overline{\lambda_k} x$, а следовательно $x \in
    E(\overline{\lambda_k}, A^*)$.

    Обратное включение доказывается аналогично.

    Итак, $E(\lambda_1, A) = E(\lambda_1, A^*)$.
    \[ H = E(\lambda_1, A) \oplus H_1, \quad H_1 = (E(\lambda_1,A))^\perp \]

    Докажем, что $H_1$ инвариантно для $A$ (аналогично показывается для $A^*$).

    Пусть $x \in H_1,\; e\in E(\lambda_1, A)$.
    \begin{multline*}
        \langle Ax, e\rangle = \langle x, A^* e\rangle = \langle x,
        \overline{\lambda_1} e\rangle = \lambda_1 \langle x, e\rangle = 0
        \Rightarrow \\ \Rightarrow Ax \perp E(\lambda_1, A) \Rightarrow Ax \in H_1
    \end{multline*}

    Таким образом операторы $A$ и $A^*$ раскладываются в прямую сумму
    операторов:
    \[
        \begin{aligned}
            A &= \lambda_1 I_1 \oplus A_2 \\
            A^* &= \overline{\lambda_1} I_1 \oplus A_2^*
        \end{aligned}
    \]

    Из пункта (4) теоремы \ref{th:invarproperties} получаем, что
    \[ \spectrum{A} = \menge{\lambda_2, \dotsc, \lambda_m} \]

    Продолжая аналогичный процесс для операторов $A_2$ и $A_2^*$ до тех пор,
    пока не останется одна точка спектра, получаем разложения:
    \begin{align*}
        A &= \lambda_1 I_1 \oplus \dotsb \oplus \lambda_{m-1} I_{m-1} \oplus A_m
        \\
        A^* &= \overline{\lambda_1} I_1 \oplus \dotsb \oplus
        \overline{\lambda_{m-1}} I_{m-1} \oplus A_m^*
    \end{align*}
    относительно подпространств
    \begin{align*}
        H &= E(\lambda_1,A) \oplus \dotsb \oplus E(\lambda_{m-1}, A) \oplus H_m
        \\
        H &= E(\overline{\lambda_1},A) \oplus \dotsb \oplus
        E(\overline{\lambda_{m-1}}, A) \oplus H_m
    \end{align*}
    где $A_m$ и $A_m^*$ — сопряженные операторы с одной точкой спектра. $A_m$ можно представить в виде
    \[ A_m = \lambda_m I_m + Q, \]
    где $Q$ — нильпотентный либо нулевой. 
    
    Понятно, что если $x \in E(\lambda_m, A)$, то $Qx = 0$. Рассмотрим прямую
    сумму
    \[ H_m = E(\lambda_m, A) \oplus (E(\lambda_m,A))^\perp \]

    Как уже было показано, $(E(\lambda_m, A))^\perp$ — инвариантное
    подпространство. Пусть $x \in (E(\lambda_m, A))^\perp$. Тогда и $Ax \in
    (E(\lambda_m, A))^\perp$.
    \begin{gather*}
        (E(\lambda_m, A))^\perp \ni Ax = \lambda_m x + Qx,\\
        \lambda_m x \in E(\lambda_m, A), \; Qx \in (E(\lambda_m,A))^\perp
    \end{gather*}

    По определению прямой суммы $Ax$ единственным образом представимо в виде
    суммы
    \[ Ax = 0 + Qx \]

    Таким образом $\lambda_m x = 0$, откуда следует, что $x = 0$, то есть
    $H_m = E(\lambda_m, A)$, $A_m = \lambda_m I_m$, что и требовалось доказать.
\end{proofbreak}

\begin{corollaryth}
    Каждый нормальный оператор допускает разложение
    \[ A = \sum_{k=1}^m \lambda_k P_k, \]
    где $P_k^* = P_k, \; k = \overline{1,m}$. При этом
    \[ A^* = \sum_{k=1}^m \overline{\lambda_k} P_k \]
    (следует из свойств сопряженных операторов)
\end{corollaryth}

\begin{theorem}
    Пусть $A \in L(H)$ --- самосопряженный оператор в комплексном евклидовом
    пространстве. Тогда его спектр состоит из вещественных собственных значений
    и существует ортонормированный базис, составленный из собственных векторов
    оператора $A$.
\end{theorem}

\begin{proof}
    По следствию из теоремы \ref{th:normaldiagonalizable} оператор $A$
    имеет допускает разложение
    \[ A = \sum_{k=1}^m \lambda_k P_k \]
    При этом
    \[ A^* = \sum_{k=1}^m \overline{\lambda_k} P_k \]
    Так как $A = A^*$ получаем
    \[ \sum_{k=1}^m (\lambda_k - \overline{\lambda_k}) P_k = O \]
    Применим к обеим частям равенства оператор $P_j, \; j = \overline{1,m}$:
    \[ P_j \bigg( \sum_{k=1}^m (\lambda_k - \overline{\lambda_k}) P_k \bigg) = O
    \]
    Отсюда получаем
    \[ (\lambda_j - \overline{\lambda_j}) P_j = O \Rightarrow \lambda_j =
    \overline{\lambda_j} \]
    То есть $\lambda_j \in \fieldr$.

    Теперь, чтобы получить ортонормированный базис, рассмотрим разложение $H$ в
    ортогональную прямую сумму
    \[ H = E(\lambda_1, A) \oplus \dotsb \oplus E(\lambda_m, A) \]
    и выберем в каждом собственном подпространстве ортонормированный базис. Их
    объединение и будет ортонормированным базисом в $H$.
\end{proof}

\begin{theorem}
    Пусть $U \in L(H)$ --- унитарный оператор. Тогда его спектр лежит на
    единичной окружности, т. е. \[\absv{\lambda_j} = 1, \; \forall \lambda_j \in
    \spectrum{U},\] и существует ортонормированный базис, состоящий из
    собственных векторов оператора $U$.
\end{theorem}

\begin{proof}
    Пусть $U \in L(H)$ --- унитарный, то есть $U^* = U^{-1}$. Тогда
    \[ U = \sum_{k=1}^m \lambda_k P_k \]
    и
    \[ U^{-1} = \sum_{k=1} \frac{1}{\lambda_k} P_k = U^* \]
    Аналогично случаю самосопряженного оператора можно показать, что
    \[ \frac{1}{\lambda_k} = \overline{\lambda_k} \]
    а это значит, что
    \[ \lambda_k \overline{\lambda_k} = 1 \Rightarrow \absv{\lambda_k} = 1 \]

    Утверждение о базисе доказывается аналогично предыдущей теореме.
\end{proof}

\begin{definition}
    Матрица $\mathcal{A}$ называется \emph{нормальной}, если 
    \[\mathcal{A} \mathcal{A}^* = \mathcal{A}^*\! \mathcal{A} \]
\end{definition}

\begin{definition}
    Матрица $\mathcal{A}$ называется \emph{ортогональной}, если 
    \[\mathcal{A} \mathcal{A}^* = \mathcal{A}^*\! \mathcal{A} = E \]
\end{definition}
