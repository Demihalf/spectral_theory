\section{Собственные значения и собственные векторы}
Далее $X$ — конечномерное линейное пространство размерности $n$, $L(X)$ —
алгебра операторов, $I$ — тождественный оператор, $O$ — нулевой оператор.

\begin{definition}
    Оператор $A$ называется \emph{скалярным}, если он имеет вид $A = \alpha I, \,
    \alpha \in \fieldk$.
\end{definition}

\begin{definition}\label{def:diagonalizable}
    Скажем, что $A\in L(X)$ имеет \emph{простую структуру} (оператор простой структуры,
    ОПС, \emph{диагонализируемый} оператор), если существует базис $e_1,
    \dotsc, e_n$ в $X$ и такие числа $\lambda_1, \dotsc, \lambda_n \in
    \fieldk$, что имеют место равенства:
    \[ Ae_k = \lambda_k e_k, \quad k = \overline{1,n} \]
\end{definition}

Иначе говоря, для того чтобы $A$ был оператором простой структуры, необходимо и
достаточно, чтобы существовал базис, в котором матрица оператора имеет вид
\[ \mathcal{A} = \begin{pmatrix}
        \lambda_1 & 0 & \ldots & 0 \\
        0 & \lambda_2 & \ldots & 0 \\
        \vdots & \vdots & \ddots & \vdots \\
        0 & 0 & \ldots & \lambda_n
\end{pmatrix} \]

Очевидно, что оператор простой структуры обратим тогда и только тогда, когда $
\forall k = \overline{1,n} \quad \lambda_k \neq 0 $, причем обратный оператор будет
иметь вид
\[ A^{-1} e_k = \frac{1}{\lambda_k} e_k, \quad k = \overline{1,n} \]

Всякий оператор скалярного типа является оператором простой структуры.

Существуют операторы, не являющиеся ОПС.

\begin{definition}
    Ненулевой оператор $ Q \in L(X) $ называется \emph{нильпотентным}, если
    $\exists m \in \mathbb{N} \quad Q^m = O$. Наименьшее из $m$, для которых
    выполняется данное равенство, называется \emph{индексом нильпотентности}
    оператора $Q$.
\end{definition}

\begin{example}
    \begin{gather*}
        D : \polyset{n} \to \polyset{n} \\
        D\phi = \phi' \\
        D^{n+1} = O 
    \end{gather*}
    $n+1$ — индекс нильпотентности
\end{example}

\begin{theorem}\label{th:nildiag}
    Если оператор нильпотентный, то он не диагонализируем
\end{theorem}

\begin{proof}
    Пусть $ Q \in L(X) $, $Q \neq 0$, $Q^m = O$.

    Предположим противное: допустим, что $Q$ — оператор простой структуры.
    Тогда существует базис $e_1, \dotsc, e_n$ в $X$ и числа $ \lambda_1, \dotsc,
    \lambda_n \in \fieldk $ такие, что 
    \[ Q e_k = \lambda_k e_k, \quad k = \overline{1,n} \]
    $\lambda_k$ не равны одновременно нулю (в противном случае $Q$ был бы
    нулевым).
    \[ \forall k = \overline{1,n} \quad 0 = Q^m e_k = \lambda_k^m e_k \Rightarrow
    \exists k = \overline{1,n} \quad e_k = 0 \]

    Получили противоречие (в базисе не может быть нулевых векторов).
\end{proof}

\begin{definition}\label{def:eigenvalue}
    Число $\lambda_0 \in \fieldk$ называется \emph{собственным значением}
    оператора $A \in L(X)$, если существует \textbf{ненулевой} вектор $x_0 \in
    X$ такой, что 
    \[ Ax_0 = \lambda_0 x_0 \]
    При этом $x_0$ называется \emph{собственным вектором} оператора $A$, соответствующим
    $\lambda_0$.
\end{definition}

Непосредственно из определения \ref{def:eigenvalue} следует, что
$\lambda_k$ из определения \ref{def:diagonalizable} являются собственными
значениями оператора $A$, а $e_k$ — собственными векторами.

\begin{definition}
    Совокупность всех собственных значений оператора $A$ называется
    \emph{спектром} оператора:
    \[ \spectrum{A} = \menge{\lambda \in \fieldk : \exists x_0 \in X \quad x_0 \neq 0
    \wedge Ax_0 = \lambda x_0} \]
\end{definition}

Рассмотрим равенство
\vspace{-1em}
\[
    \begin{gathered}
        Ax_0 = \lambda_0 x_0 \\
        (A - \lambda_0 I)x_0 = 0 
    \end{gathered}
\]

Если $ \ker (A - \lambda_0 I) \neq \menge{0} $, то $\lambda_0$ — собственное значение
$A$. Значит, можно сказать, что

\emph{Спектр оператора — множество тех $\lambda_0 \in \fieldk$, для которых оператор
$A - \lambda_0 I$ необратим.}

Оператор $A - \lambda_0 I$ необратим тогда и только тогда, когда 
\begin{equation} \label{eq:invertibility}
    \det (A-\lambda_0 I) = 0 
\end{equation}

Пусть $\mathcal{A} = (a_{ij}) \in \matr{n}$ — матрица оператора $A$. Исходя из
условия (\ref{eq:invertibility}), $\lambda$ является собственным значением
оператора $A$ тогда и только тогда, когда матрица $\mathcal{A} - \lambda E$
необратима, т. е. её определитель равен нулю:
\begin{gather} 
    \det (a_{ij} - \lambda \delta_{ij}) = 0 \nonumber \\[0.5em]
    p_A(\lambda) = \begin{vmatrix}
            a_{11} - \lambda & a_{12} & \ldots & a_{1n} \\
            a_{21} & a_{22} - \lambda & \ldots & a_{2n} \\
            \vdots & \vdots & \ddots & \vdots \\
            a_{n1} & a_{n2} & \ldots & a_{nn} - \lambda
    \end{vmatrix} \label{eq:charpolynomial}
\end{gather}

\begin{equation} \label{eq:charpolynomialexp}
    p_A(\lambda) = (-1)^{n}\lambda^n + (-1)^{n-1} (a_{11} + \dotsb +
    a_{nn}) \lambda^{n-1} + \dotsb + \det \mathcal{A}
\end{equation}

\begin{definition}
    Выражение (\ref{eq:charpolynomial}) называется \emph{характеристическим
    многочленом} оператора $A$ (матрицы $\mathcal{A}$).
\end{definition}

Из вышеприведённых рассуждений следует
\begin{theorem} \label{th:charpolynomial}
    Спектр оператора состоит из корней характеристического многочлена:
    \[ \spectrum{A} = \menge{\lambda \in \fieldk : p_A(\lambda) = 0 } \]
\end{theorem}

Следовательно, спектр линейного оператора в конечномерном линейном пространстве
всегда содержит не более $n$ собственных значений.
\begin{example}
    Рассмотрим $\fieldr^2$ и оператор $A \in L(\fieldr^2)$ с матрицей~$\mathcal{A}$:
    \begin{gather*}
        \mathcal{A} = \begin{pmatrix}
                0 & 1 \\
                -1 & 0
        \end{pmatrix}, \quad
        A(x_1, x_2) = (x_2, -x_1) \\
        \det \begin{pmatrix}
                -\lambda & 1 \\
                -1 & -\lambda
        \end{pmatrix} = \lambda^2 + 1 = 0
    \end{gather*}

    Вещественных корней нет, значит
    \[ \spectrum{A} = \varnothing \]
\end{example}

\begin{theorem} \label{th:complexspectrum}
    Пусть $X$ — комплексное ЛП, $\dim X \geq 1$. Тогда каждый $A\in L(X)$ имеет
    непустой спектр, состоящий из конечного числа собственных значений, число
    которых не превышает $n$, и все они совпадают с корнями $p_A(\lambda)$.
\end{theorem}

Доказательство теоремы следует из приведённых выше рассуждений и основной
теоремы высшей алгебры.

\begin{definition}
    \emph{Следом} квадратной матрицы называется сумма элементов матрицы,
    стоящих на главной диагонали.
    \[ \tr \mathcal{A} = \sum_{i=1}^n a_{ii} \] 
\end{definition}

Аналогично определяется след оператора (можно показать, что подобные матрицы
имеют одинаковый след).

Из формул Виета и выражения (\ref{eq:charpolynomialexp}) получаем
\begin{align*}
    \lambda_1 + \lambda_2 + \dotsb + \lambda_n &= a_{11} + a_{22} + \dotsb +
    a_{nn} = \tr A \\
    \lambda_1 \lambda_2 \dotsm \lambda_n &= \det \mathcal{A} = \det A
\end{align*}

\begin{definition}
    Ядро оператора $A - \lambda_0 I$ называется \emph{собственным
    подпространством} оператора $A$ для собственного значения $\lambda_0$.
    \[ E(\lambda_0, A) = \ker (A - \lambda_0 I) \]
\end{definition}

\begin{definition}
    Кратность корня $\lambda_0$ характеристического многочлена оператора $A$
    называется \emph{алгебраической кратностью} собственного значения
    $\lambda_0$.
\end{definition}

\begin{definition}
    Собственное значение линейного оператора называется \emph{простым}, если его
    алгебраическая кратность равна единице.
\end{definition}

\begin{definition}
    $\dim E(\lambda_0, A)$ называется \emph{геометрической кратностью} собственного
    значения $\lambda_0$.
\end{definition}

\begin{theorem} \label{th:linindependenteigenvectors}
    Пусть $\lambda_1, \dotsc, \lambda_m \in \fieldk$ — различные собственные
    значения оператора $A\in L(X)$. Тогда соответствующие им собственные векторы
    $e_1, \dotsc, e_m$ линейно независимы.
\end{theorem}
\begin{proof}
    Доказательство проведём индукцией по $m$.
    
    \emph{База индукции:} при $ m=1$ утверждение теоремы очевидно.

    \emph{Индукционный переход:} пусть утверждение верно для $k$ векторов.
    $e_1, \dotsc, e_k$ — линейно независимы. Добавим к ним вектор $e_{k+1}$.
    \begin{gather*}
        \sum_{j=1}^{k+1} \alpha_j e_j = 0 \Rightarrow
        (A - \lambda_{k+1} I)\bigg( \sum_{j=1}^{k+1} \alpha_j e_j \bigg) = 0 \\
        \sum_{j=1}^{k} (\lambda_j - \lambda_{k+1}) \alpha_j e_j = 0 \Rightarrow
        \alpha_j = 0, \quad j = \overline{1,k}  \\
        \alpha_{k+1} e_{k+1} = 0 \Rightarrow \alpha_{k+1} = 0, \text{т.к. }
            e_{k+1} \neq 0 
    \end{gather*}
\end{proof}

Из определения \ref{def:diagonalizable} и теоремы \ref{th:linindependenteigenvectors} следует

\begin{theorem} \label{th:simpledifferenteigenvalues}
    Если характеристический многочлен оператора имеет $n$ различных корней и
    размерность пространства равна $n$, то этот оператор диагонализируем.
\end{theorem}
