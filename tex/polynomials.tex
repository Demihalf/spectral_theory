\section{Многочлены от операторов и матриц}
Далее будем рассматривать алгебры линейных операторов, матриц и
многочленов.

Пусть $A \in L(X)$, $p \in \polyset{}$.
\[ p(z) = p_0 + p_1 z + \dotsb + p_m z^m \]

Поставим $p$ в соответствие такой оператор $p(A)$, что
\[ p(A) = p_0 I + p_1 A + \dotsb + p_m A^m \]

\begin{definition}
    Оператор $p(A)$ называется \emph{многочленом от оператора} $A$.
\end{definition}

Рассмотрим отображение $ \Phi_A : \polyset{} \to L(X) $ такое что
\begin{equation}\label{eq:operatorpolynomial}
    \Phi_A(p) = p(A) 
\end{equation}

\begin{lemma}
    Отображение \eqref{eq:operatorpolynomial} — гомоморфизм алгебр:
    \begin{enumerate}
        \item $\Phi_A(\mathbbm{1}) = I$, где $\mathbbm{1}(z) \equiv 1$
        \item $\Phi_A(\alpha f + \beta g) = \alpha \Phi_A(f) + \beta \Phi_A(g)$
        \item $\Phi_A(fg) = \Phi_A(f) \Phi_A(g)$
    \end{enumerate}
\end{lemma}

\begin{proofbreak}
    \NoEndMark
    \dindent Утверждения (1) и (2) очевидны. Докажем утверждение (3).
    \begin{enumerate}
        \setcounter{enumi}{2}
        \item $
            \begin{gathered}[t]
                \begin{aligned}[t]
                    f(z) &= f_0 + f_1 z + \dotsb + f_m z^m \\
                    g(z) &= g_0 + g_1 z + \dotsb + g_k z^k \\
                    (fg)(z) &= \sum_{i=0}^{m+k} c_i z^i, & c_i = \sum_{j+p=i} f_j
                        g_p \\
                    \Phi_A(fg) &= \sum_{i=0}^{m+k} c_i A^i
                \end{aligned} \\
                \begin{multlined}
                    \Phi_A(f) \Phi_A(g) = f(A)g(A) = (f_0 I + f_1 A + \dotsb 
                        + f_m A^m) \times \\ \times (g_0 I + g_1 A
                        + \dotsb + g_k A^k) = \sum_{i=0}^{m+k} c_i A^i, \quad c_i =
                        \sum_{j+p=i} f_j g_p
                \end{multlined} \\
                \Phi_A(fg) = \Phi_A(f) \Phi_A(g) \qquad\square
            \end{gathered} $
    \end{enumerate}
\end{proofbreak}

\begin{corollaryle}
    $ f(A) g(A) = g(A) f(A) $
\end{corollaryle}

\begin{proof}
    \[ \Phi_A(f) \Phi_A(g) = \Phi_A(fg) = \Phi_A(gf) = \Phi_A(g) \Phi_A(f) \] 
\end{proof}

Аналогично можно рассмотреть многочлены от матриц (и вообще элементов любой
унитарной алгебры).

Несложно заметить, что многочлен $p(\mathcal{A})$ от матрицы $\mathcal{A}$ 
оператора $A$ есть матрица многочлена $p(A)$ от оператора $A$.

Рассмотрим оператор $A\in L(X)$ такой, что
\begin{align*}
    I &= I_1 \oplus \dotsb \oplus I_m \\
    A &= A_1 \oplus \dotsb \oplus A_m \\
    \intertext{относительно}
    X &= X_1 \oplus \dotsb \oplus X_m
\end{align*}

Очевидно, что если $X_k$ инвариантно относительно $A$, то оно инвариантно также
относительно $A^j, \; j \geq 0$. Также ясно, что если $A_k$~— сужение оператора
$A$ на инвариантное подпространство $X_k$, то $A_k^j$ — сужение оператора $A^j$
на то же подпространство. Отсюда получаем, что
\[ A^j = A_1^j \oplus \dotsb \oplus A_m^j \]

Пусть далее $A$ — оператор простой структуры, $f\in \polyset{}$.
\begin{align*}
    A &= \sum_{j=1}^m \lambda_j P_j \\
    A^k &= \sum_{j=1}^m \lambda_j^k P_j \\
    f(A) &= \sum_{j=1}^m f(\lambda_j) P_j
\end{align*}

Последние два равенства доказываются простой проверкой.

Наша цель — получить из последнего равенства проекторы $P_j$. Этого можно
добиться, если найти такие многочлены $f_j$, что
\[
\left\{\begin{array}{l l}
    f_j(\lambda_j) = 1 \\
    f_j(\lambda_k) = 0, & k \neq j
\end{array}\right.
\]

Тогда $f_j(A) = P_j$.

Используем интерполяционную формулу Лагранжа:
\begin{gather}
    f_j(\lambda) = \prod_{\substack{i = 0 \\ i \neq j}}^m \frac{\lambda -
        \lambda_i}{\lambda_j - \lambda_i} \nonumber \\ 
    A = \sum_{j=1}^m \lambda_j P_j = \sum_{j=1}^m \lambda_j f_j(A) =
    \sum_{j=1}^m \lambda_j \prod_{\substack{i = 0 \\ i \neq j}}^m \frac{A -
    \lambda_i}{\lambda_j - \lambda_i} \label{eq:sylvester}
\end{gather}

Формула \eqref{eq:sylvester} называется \emph{интерполяционной формулой
Сильвестра}.
