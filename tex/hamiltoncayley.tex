\section{Теорема Гамильтона-Кэли}
\begin{definition}
    Ненулевой многочлен $p \in \polyset{}$ называется \emph{аннулирующим} оператор $A
    \in L(X)$ (матрицу $\mathcal{A} \in \matr{n}$), если $p(A) = O$
    ($p(\mathcal{A}) = 0$).
\end{definition}

\begin{definition}
    Многочлен $p$ наименьшей положительной степени со старшим коэффициентом $1$
    и аннулирующий оператор $A$ называется \emph{минимальным (аннулирующим) многочленом}
    оператора $A$.
\end{definition}

Рассмотрим без доказательства следующую важную теорему.

\begin{theorem}[Гамильтона-Кэли] \label{th:hamiltoncayley}
    Пусть $X$ — линейное пространство над полем комплексных чисел.
    Тогда характеристический многочлен оператора $A \in L(X)$ аннулирует оператор
    $A$.
    \[ p_A(A) = O \]
\end{theorem}

\begin{lemma}
    Минимальный многочлен для оператора единственен.    
\end{lemma}

\begin{proof}
    Предположим противное. Пусть $f$ и $g$ — минимальные многочлены оператора $A$:
    \[ f(A) = g(A) = O \]

    Тогда
    \[ (f - g)(A) = O \]

    То есть многочлен $f - g$ также аннулирующий, причем его степень меньше
    степени $f$ и $g$. Получили противоречие минимальности $f$ и $g$.
\end{proof}

\begin{corollaryth}[из теоремы \ref{th:hamiltoncayley}]
    Пусть $A\in L(X)$, $\spectrum{A} = \menge{\lambda_0}$. Тогда
    \[ A = \lambda_0 I + Q, \]
    где $Q$ — нильпотентный либо нулевой оператор и $Q^n = O$, $n = \dim X$
\end{corollaryth}

\begin{proof}
    Характеристический многочлен оператора~$A$ имеет вид
    \[ p_A(\lambda) = (-1)^n (\lambda-\lambda_0)^n \]

    Тогда по теореме Гамильтона-Кэли получаем
    \[ O = p_A(A) = (A-\lambda_0 I)^n \]

    Таким образом оператор $A-\lambda_0 I = Q$ — нильпотентный и 
    \[ A = \lambda_0 I + Q \]
\end{proof}
